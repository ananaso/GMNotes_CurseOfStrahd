\pagebreak
\subsection{Dungeon Level Access}
\label{sec:DungeonLevelAccess}
\begin{readout}
  The wooden spiral staircase from the attic ends in a narrow tunnel that stretches southward before branding east
  and west. The tunnel appears to be carved directly out of earth, clay, and rock. The tunnels are 4 feet wide by
  7 feet high with timber braces at 5-foot intervals. The tunnel is absolutely dark, almost to the point where
  it feels like the shadows are pressing in towards you.
  
  As you reached the bottom of the stairs and started moving into the tunnel, you began to hear an eerie, incessant
  chant echoing throughout.
\end{readout}
It's impossible to gauge where the sound is coming until the PCs reach the Hidden Spike Pit (Area 26) or Ghoulish
Encounter (Area 29). They can't discern its words until they reach the Reliquary (Area 35).

If PCs have light or darkvision as they begin to move into the tunnel:
\begin{readout}
  As you begin to explore the dungeon, you see seemingly centuries-old human footprints in the earthen floor
  leading every which way.
\end{readout}

\begin{arealinks}
  \arealink{sec:SecretStairs}
  \arealink{sec:FamilyCrypts}
  \arealink{sec:CultInitiatesQuarters}
  \arealink{sec:DiningHall}
\end{arealinks}


\pagebreak
\subsection{Family Crypts}
\label{sec:FamilyCrypts}
Several crypts have been hewn from the earth. Each crypt is sealed with a stone slab unless noted otherwise.
Removing a slab from its fitting requires a successful DC 15 Strength (Athletics) check; using a crowbar or
the like grants advantage on the check.

\subsubsection{Empty Crypt}
\begin{readout}
  The blank stone slab meant to seal this crypt leans against a nearby wall. The crypt is empty.
\end{readout}

\subsubsection{Walter's Crypt}
\begin{readout}
  The stone slab meant to seal this crypt leans against a nearby wall. Etched into it is the name
  ``Walter Durst''. The crypt is empty.
\end{readout}

\subsubsection{Gustav's Crypt}
\begin{readout}
  The stone slab is etched with the name ``Gustav Durst''.
  
  \textit{If PCs open the crypt:} The chamber beyond contains a coffin atop a stone bier.
  
  \textit{If PCs open/disturb the coffin:} The coffin is empty.
\end{readout}

\subsubsection{Elisabeth's Crypt}
\begin{readout}
  The stone slab is etched with the name ``Elisabeth Durst''.
  
  \textit{If PCs open the crypt:} The crypt contains a stone bier with a coffin atop it.
  
  \textit{If PCs open/disturb the coffin:} A swarm of centipedes boils out of the back wall and attacks!
\end{readout}

\subsubsection{Rose's Crypt}
\begin{readout}
  The stone slab is etched with the name ``Rosavalda Durst''.
  
  \textit{If PCs open the crypt:} The chamber beyond contains a coffin on a stone bier.
  
  \textit{If PCs open the coffin:} The coffin is empty.
  
  \textit{If PCs place Rose's skeletal remains in the coffin:} You hear a light sigh on the air as well as a
  faint ``Thank you!'' in Rose's voice; she sounds relieved. \{\textit{If a PC is possessed by Rose:} You feel 
  Rose's presence leave you; you are no longer possessed.\}
\end{readout}

\subsubsection{Thorn's Crypt}
\begin{readout}
  The stone slab is etched with the name ``Thornboldt Durst''.
  
  \textit{If PCs open the crypt:} The chamber beyond contains a coffin on a stone bier.
  
  \textit{If PCs open the coffin:} The coffin is empty.
  
  \textit{If PCs place Thorn's skeletal remains in the coffin:} You hear a light woosh as well as a faint,
  almost upbeat or happy ``Bye!'' in Thorn's voice. \{\textit{If a PC is possessed by Thorn:} You feel Thorn's
  presence leave you; you are no longer possessed.\}
\end{readout}

\begin{arealinks}
  \arealink{sec:DungeonLevelAccess}
  \arealink{sec:CultInitiatesQuarters}
  \arealink{sec:DiningHall}
\end{arealinks}


\pagebreak
\subsection{Cult Initiates' Quarters}
\label{sec:CultInitiatesQuarters}
\begin{readout}
  A wooden table and four chairs stand at the east end of this room. The room is 8 feet tall and supported by
  thick wooden posts with crossbeams. To the west are four alcoves containing moldy pallets.
\end{readout}

\begin{arealinks}
  \arealink{sec:DungeonLevelAccess}
  \arealink{sec:FamilyCrypts}
  \arealink{sec:WellAndCultistQuarters}
\end{arealinks}


\pagebreak
\subsection{Well and Cultist Quarters}
\label{sec:WellAndCultistQuarters}
\begin{readout}
  A 4-foot-diameter well shaft with a 3-foot-high stone lip descends 30 feet to a water-filled cistern.
  A wooden bucket hangs from a rope-and-pulley mechanism bolted to the crossbeams above the well. There are
  five siderooms that each contain a wood-framed bed with a moldy straw matress and a wooden chest, presumably
  for personal belongings. Each chest is secured with a rusty iron padlock.
\end{readout}
There is a skeleton (unanimated) at the bottom of the well. The padlocks can all be picked with thieves' tools 
and a successful DC 15 Dexterity check.

\subsubsection*{Treasure}
In addition to some worthless personal effects, each chest contains one or more valuable items.
\begin{enumerate}[label={\textit{25\Alph*.}}]
  \item This room's chest contains 11 gp and 60 sp in a pouch made of human skin.
  \item This room's chest contains three moss agates (worth 10 gp each) in a folded piece of black cloth.
  \item This room's chest contains a black leather eyepatch with a carnelian (worth 50 gp) sewn into it, along
      with a logbook bound in grimy black leather. This journal maintains a list of names, physical
      descriptions, and details of some sort of event. The details in this third column are rather gruesome,
      detailing what happened to each listed person in terms such as ``struggled profusely'' and ``no sedative 
      given''; from what you can glean, these appear to be describing how the persons listed were sacrificed
      for some unknown purpose.
      
      \{\textit{GM Note:} The journal was kept by one of the head cultists as a record of the cult's victims.\}
  \item This room's chest contains an ivory hairbrush with silver bristles (worth 25 gp).
  \item This room's chest contains a silvered shortsword (worth 110 gp).
\end{enumerate}

\begin{arealinks}
  \arealink{sec:CultInitiatesQuarters}
  \arealink{sec:HiddenSpikePit}
  \arealink{sec:DiningHall}
\end{arealinks}

\subsubsection*{Skill Challenge Escape: Well \& Cultist Quarters}
\label{sec:SC_WellAndCultistQuarters}
\begin{readout}
  This room is heavily obscured by an unnatural black fog.
\end{readout}
As the PCs cross the room to leave the chamber, if they haven't already succeeded on a check in this room, a
skeleton reaches out of the well and grapples the last PC in the marching order. On a failure, the adventurer
is nearly pulled into the well where they struggle against the skeleton that seeks to drown them; ultimately,
the victim escapes, but not worse for wear.
\begin{skillChallenge}
  \begin{itemize}
    \item \textbf{Acrobatics} or \textbf{Athletics} can be used to break the grapple. \easyDC
    \item \textbf{Insight} can be used to recall one's steps, if the PCs explored this room beforehand. \easyDC
    \item \textbf{Perception} can be used to navigate the darkness, hear the skeleton before it strikes, or
    find its victims. \moderateDC
  \end{itemize}
\end{skillChallenge}


\pagebreak
\subsection{Hidden Spike Pit}
\label{sec:HiddenSpikePit}
\begin{readout}
  The ghostly chanting heard throughout the dungeon gets discernibly louder as you turn down this tunnel. 
\end{readout}
Pull the first player to enter this hallway into a breakout room, and continue with:
\begin{readout}
  As you turn the corner, something strikes you as odd about this hallway but you can't quite put your finger
  on it.
  
  \{\textit{Have them roll against a DC 15 Wisdom (Perception) check.}\}
  
  \textit{Success:} You notice that there is a distinct absence of footprints in this hallway. This is in stark
  contrast to all the other tunnels you've been in down here, which had plenty of very noticeable centuries-old
  footprints leading every which way.
  
  \textit{Failure:} You don't detect anything, and the feeling passes. It must have just been in your head,
  perhaps something about these tunnels making you a bit more paranoid than you normally would be.
\end{readout}
If PCs search for traps:
\begin{readout}
  You find a 5-foot-long, 10-foot-deep pit hidden under several rotted wooden planks, all hidden under a thin
  layer of dirt. The pit has sharpened wooden spikes at the bottom.
\end{readout}
The first PC to step on the cover falls through, landing prone and taking 1d6 bludgeoning damage from the fall
plus 2d10 piercing damage from the spikes.

\begin{arealinks}
  \arealink{sec:WellAndCultistQuarters}
  \arealink{sec:DiningHall}
  \arealink{sec:GhoulishEncounter}
  \arealink{sec:StairsDown}
\end{arealinks}

\subsubsection*{Skill Challenge Escape: Hidden Spike Pit}
\label{sec:SC_HiddenSpikePit}
There is no mandatory obstacle here, but a trap that might not have been previously encountered by the
adventurers; if so, run it as-is above. If nobody either notices the trap or if they don't alert the rest of
the party, run it as an obstacle.
\begin{skillChallenge}
  \begin{itemize}
    \item \textbf{Sleight of Hand} can be used to snatch at another falling PC's belt, or snatch at the edge
    if preventing their own fall. \easyDC
    \item \textbf{Acrobatics} may allow a PC to divert their momentum into a leap, landing safely on the opposite
    side. \moderateDC
  \end{itemize}
\end{skillChallenge}
Once the hidden spike pit is known to the party, it's an automatic success to cross it.


\pagebreak
\subsection{Dining Hall}
\label{sec:DiningHall}
\begin{readout}
  This room contains a plain wooden table flanked by long benches. Moldy humanoid bones lie strewn on the dirt
  floor. In the middle of the south wall is a darkened alcove.
\end{readout}
The bones are the remains of the cult's vile banquets. PCs who approach within 5 feet of the alcove (Area 28)
provoke the creature that lurks there.

\begin{arealinks}
  \arealink{sec:DungeonLevelAccess}
  \arealink{sec:FamilyCrypts}
  \arealink{sec:WellAndCultistQuarters}
  \arealink{sec:HiddenSpikePit}
  \arealink{sec:Larder}
  \arealink{sec:GhoulishEncounter}
\end{arealinks}

\subsubsection*{Skill Challenge Escape: Dining Hall}
\label{sec:SC_DiningHall}
\begin{readout}
  Screams rend the darkened depths. Screams for mercy, for help, for a quick end. You come across a man chained
  to the wooden table, thrashing, screaming. A gash runs the length of his belly, from which blood pulses out to
  the beat of his heart! How or where he came from doesn't matter, because in the distance you can hear them:
  the cultists, chanting, hungering! It sounds like they are quickly approaching and will be upon you soon.
\end{readout}
A ghost of Death House's red past has been made flesh once again, and mad babbling threatens to draw the
ghostly cultists upon the adventurers. He has the statistics of a restrained commoner with 1 hit point remaining
and is bound by chains.

If the PCs linger here, five cultists (shadows) arrive in 2 rounds and descend upon the man if he yet remains.
If slain, he does not die quietly.
\begin{skillChallenge}
  \begin{itemize}
    \item \textbf{Athletics} can be used with a weapon to break the man's chains \hardDC, while
    \textbf{thieves' tools} can unlock them. \moderateDC
    \item \textbf{Deception} or \textbf{Persuasion} can be used to deceive the man into calm. \moderateDC
    \item \textbf{Medicine} can be used to dress his wounds, if the PC has a healing kit with 1 action.
    \moderateDC
    \item \textbf{Spells} that restore hit points (healing word, cure wounds) can be used to heal the
    screaming man. (Automatic Success)
  \end{itemize}
\end{skillChallenge}
If the PCs succeed here, this lone spirit will tell them of the secret trapdoor to the first floor (Area 32)
and lead them to it if he's been freed.


\pagebreak
\subsection{Larder}
\label{sec:Larder}
This alcove contains a grick that slithers out to attack the first character it sees within 5 feet of it.
Any character with a passive Wisdom (Perception) score under 12 is surprised by it. The alcove is otherwise
empty.

\begin{arealinks}
  \arealink{sec:DiningHall}
\end{arealinks}


\pagebreak
\subsection{Ghoulish Encounter}
\label{sec:GhoulishEncounter}
\begin{readout}
  The ghostly chanting heard throughout the dungeon is noticeably louder to the north. However, you're struck
  by an overpowering stench of death and decay coming from down the hallway.
\end{readout}
When one or more PCs reach the midpoint of the four-way tunnel intersection:
\begin{readout}
  As you walk into the center of this intersection, a decaying, clawed hand bursts out of the ground and
  grips your foot like a vice.
\end{readout}
Have the affected PC roll a Strength (Athletics) or Dexterity (Acrobatics) check against a ghoul's Strength
(Athletics) check. If they succeed, they're able to take a surprise round. Otherwise, combat begins normally.

Two ghouls (reduced from four to make this simply hard rather than deadly, per Kobold Fight Club) rise up
out of the ground in the two closest spaces marked X on the map and attack. The ghouls fight until destroyed.

As the ghouls are the undead forms of the former cultists, they retain some vestige of their former selves.
They mindlessly repeat any or all of the following phrases as they attack the PCs: ``Beautiful. We're so
beautiful''; ``Nothing can hurt us''; ``We are perfect. We are immortal''; and ``Help us live forever''.

\begin{arealinks}
  \arealink{sec:HiddenSpikePit}
  \arealink{sec:DiningHall}
  \arealink{sec:StairsDown}
  \arealink{sec:DarklordsShrine}
  \arealink{sec:CultLeadersDen}
\end{arealinks}


\pagebreak
\subsection{Stairs Down}
\label{sec:StairsDown}
\begin{readout}
  You approach a 20-foot-long flight of stairs leading further down into the dungeon. It's obvious that the
  ghostly chants originate from somewhere below.
\end{readout}

\begin{arealinks}
  \arealink{sec:HiddenSpikePit}
  \arealink{sec:GhoulishEncounter}
  \arealink{sec:Reliquary}
\end{arealinks}


\pagebreak
\subsection{Darklord's Shrine}
\label{sec:DarklordsShrine}
\begin{readout}
  This room is festooned with moldy skeletons that hang from rusty shackles against the walls. A wide alcove
  in the south wall contains a painted wooden statue carved in the likeness of a gaunt, pale-faced man wearing
  a voluminous black cloak, his pale left hand resting on the head of a wolf that stands next to him. In his
  right hand, he holds a smoky-gray crystal orb. There are five ashen shadows burned into the walls, with soot
  marks stretching across the floor toward the statue.
  
  The room has exits in the west and north walls. Chanting can be heard coming from the west.
\end{readout}
The statue depicts Strahd, to whom the cultists made sacrifices in the vain hope that he might reveal his
darkest secrets to them. The skeletons on the wall are harmless decor.

If a PC approaches the statue:
\begin{readout}
  As you approach the statue, you can hear many voices whispering: ``His gaze burns upon us''; ``The Darklord's
  eyes are always watching''; and ``He is the Ancient; He is the land''
\end{readout}

If the characters disturb the crystal orb in Strahd's hand:
\begin{readout}
  The five shadows begin swooping across the walls menacingly. You hear murmured moans, including phrases
  such as ``Begone from this place!'' and ``Look not upon us!''.
\end{readout}
If the PCs leave without taking the orb or moving it from Strahd's hand, the shadows don't attack. Otherwise, 
two shadows attack after 1 round, two shadows the round after that, and the final shadow a round after the
second pair of shadows. The shadows (the spirits of former cultists) pursue those who flee beyond the room's
confines.

\subsubsection*{Concealed Door}
PCs searching the room for secret doors find a concealed door in the middle of the east wall with a successful
DC 10 Wisdom (Perception) check.
\begin{readout}
  You discover an ordinary -- albeit rotted -- wooden door hidden under a layer of clay. The door pulls open
  to reveal a stone staircase that climbs 10 feet to a landing.
\end{readout}
The staircase goes to the Hidden Trapdoor (Area 32).

\subsubsection*{Treasure}
The crystal orb is worth 25 gp. It can be used as an arcane focus but is not magical.

\begin{arealinks}
  \arealink{sec:GhoulishEncounter}
  \arealink{sec:HiddenTrapdoor}
  \arealink{sec:CultLeadersDen}
\end{arealinks}


\pagebreak
\subsection{Hidden Trapdoor}
\label{sec:HiddenTrapdoor}
\begin{readout}
  The staircase ends at a landing with a 6-foot-high ceiling of close-fitting planks with a wooden trapdoor
  set into it. The trapdoor is bolted shut from this side
\end{readout}
If the trapdoor is pushed open, it reveals the den (Area 3) above.

\subsubsection*{Development}
Once the trapdoor has been found and opened, it remains available to PCs as a way into and out of the
dungeon level.

\begin{arealinks}
  \arealink{sec:DenOfWolves}
  \arealink{sec:DarklordsShrine}
\end{arealinks}


\pagebreak
\subsection{Cult Leaders' Den}
\label{sec:CultLeadersDen}
\begin{readout}
  A chandelier is suspended above a table in the middle of the room. Two high-backed chairs flank the table,
  which has an empty clay jub and two clay flagons atop it. Iron candlesticks stand in two corners, their
  candles long since melted away.
\end{readout}
\{\textit{I've removed the mimic since it doesn't fit the theme - mimics don't show up anywhere else in CoS. I'm 
leaving the text here just in case I do want to run it.}
The door in the southwest is a mimic in disguise. Any creature that touches the door becomes adhered to the
creature, whereupon the mimic attacks. The mimic also attacks if it takes any damage; if it's attacked at
range by a wary or alerted PC, however, it flees, vanishing around the corner and reappearing as a door,
chest, or longsword elsewhere in the dungeon.\}

\begin{arealinks}
  \arealink{sec:GhoulishEncounter}
  \arealink{sec:DarklordsShrine}
  \arealink{sec:CultLeadersQuarters}
\end{arealinks}


\pagebreak
\subsection{Cult Leaders' Quarters}
\label{sec:CultLeadersQuarters}
\begin{readout}
  This room contains a large wood-framed bed with a rotted feather mattress, a wardrobe containing several old
  robes, a pair of iron candlesticks, and an open crate containing thirty torches and a leather sack with
  fifteen candles inside it. At the foot of the bed is a wooden footlocker. The east wall, facing the footlocker,
  has crumbled revealing an empty human-sized alcove.
\end{readout}
When a PCs have removed the items from the chest:
\begin{readout}
  As you finish rifling through the footlocker, a ghast suddenly bursts out from the north wall! She is wearing
  a tattered, once-beautiful red dress, gold earrings, and a golden necklace around her neck. Her lips and
  gums have gone black with rot, and her smile shines with madness. The ghast bears a vague resemblence to the
  woman you've seen pictured throughout the house, but is terribly disfigured and twisted into a horrific,
  ghastly form. \{\textit{GM Note: Get it? ``Ghast-ly!''}\}
\end{readout}
Unlike the ghouls, Mrs. Durst retains the ability of speech and her memory, but has completely succumbed to
her own dark whims and is completely insane. She is arrogant to an extreme and shuns her dead husband, calling
him a lecherous traitor who deserved his death. She speaks unkindly of Walter and the nursemaid, and even writes
off Rose and Thorn as ``bothersome nuisances''. She is vulgar to a fault and speaks in a hissing, gurgling voice.

Should the players ask her what she did to Walter, she cackles, grins, and invites them to descend further into 
the basement to ``see for themselves''.

If reduced to half hitpoints, Mrs. Durst defensively backs herself into the corner and commands the PCs to leave.

\subsubsection*{Treasure}
\begin{readout}
  The footlocker is unlocked and contains the following gear and magic items:
  \begin{itemize}
    \item Folded cloak of protection
    \item Small unlocked wooden coffer containing four potions of healing
    \item Chain shirt
    \item Mess kit
    \item Flask of alchemist's fire
    \item Bullseye lantern
    \item Set of thieves' tools
    \item Spellbook with a yellow leather cover
  \end{itemize}
\end{readout}
The spellbook contains the following wizard spells:
\begin{itemize}
  \item \nth{1} Level:
  \begin{itemize}
    \item Disguise Self
    \item Identfy
    \item Mage Armor
    \item Magic Missile
    \item Protection from Evil and Good
  \end{itemize}
  \item \nth{2} Level:
  \begin{itemize}
    \item Darkvision
    \item Hold Person
    \item Invisibility
    \item Magic Weapon
  \end{itemize}
\end{itemize}
These items were taken from adventurers who were drawn into Barovia, captured, and killed by the cult.

\begin{arealinks}
  \arealink{sec:CultLeadersDen}
\end{arealinks}


\pagebreak
\subsection{Reliquary}
\label{sec:Reliquary}
\begin{readout}
  The ghostly chant fills this room and seems to be emanating from the west. You can discern a dozen or so
  voices saying, over and over, ``He is the Ancient. He is the land.''
  
  Around the room are thirteen niches dug into the walls, with each containing an item of questionable
  value and provenance.
\end{readout}
Items include:
\begin{itemize}
  \item An angelic feather
  \item A knife carved from a human bone
  \item A dagger with a rat's skull set into the pommel
  \item A pile of severed raven talons
  \item An aspergillum carved from bone
  \item A folded clock made from stitched ghoul skin
  \item A desiccated frog lashed to a stick (could be mistaken for a wand of polymorph)
  \item A bag full of bat guano
  \item A hag's severed finger
  \item A cracked egg containing the remains of a skeletal infant dragon
  \item An iron pendant adorned with a devil's face on the front an the emblem of a rose on the back
  \item A small chunk of amber resin that exudes an evil aura
  \item A small wooden coffer containing a dire wolf's withered tongue
\end{itemize}
The southernmost tunnel slopes down at a 20-degree angle into murky water and ends at a rusty portcullis
(Area 37).

\begin{arealinks}
  \arealink{sec:StairsDown}
  \arealink{sec:Prison}
  \arealink{sec:Portcullis}
\end{arealinks}


\pagebreak
\subsection{Prison}
\label{sec:Prison}
\begin{readout}
  This appears to be a prison, given that there are rusty shackles against the back wall of each alcove you
  can see.
\end{readout}

\subsubsection*{Secret Door}
A secret door in the south wall can be found with a successful DC 15 Wisdom (Perception) check that pulls
open to reveal area 38 beyond.

\subsubsection*{Treasure}
If the players approach the cell marked X on the map:
\begin{readout}
  Hanging on the back wall of this cell is a human skeleton clad in a tattered black robe, and hanging around 
  its neck is a wooden sign that reads ``unbeliever'' written in what appears to be dried blood.
\end{readout}
The skeleton belongs to a cult member who questioned the cult's blind devotion to Strahd. PCs who search the
skeleton find a gold ring (worth 25 gp) on one of its bony fingers.

\begin{arealinks}
  \arealink{sec:Reliquary}
  \arealink{sec:RitualChamber}
\end{arealinks}


\pagebreak
\subsection{Portcullis}
\label{sec:Portcullis}
\begin{readout}
  This tunnel is blocked by a rusty iron portcullis submerged in 2 feet of murky water. There is a wooden
  wheel half-embedded in the wall by the gate that looks like it would open the way.
\end{readout}
The wooden wheel raises and lowers the portcullis. The portcullis can be forcibly lifted with a successful
DC 20 Strength (Athletics) check.

\begin{arealinks}
  \arealink{sec:Reliquary}
  \arealink{sec:RitualChamber}
\end{arealinks}


\pagebreak
\subsection{Ritual Chamber}
\label{sec:RitualChamber}
The cult used to perform rituals in this sunken room. The chanting heard throughout the dungeon originates here,
yet when the PCs arrive the dungeon falls silent as the chanting mysteriously stops.
\begin{readout}
  The chanting stops as you peer into this forty-foot-square room and it's suddenly eerily quiet. The smooth
  masonry walls provide excellent acoustics. Featureless stone pillars support a vaulting ceiling, and a breach
  in the west wall leads to a dark cave heaped with refuse. Murky water covers most of the floor. Stairs
  lead up to dry stone ledges that hug the walls. In the middle of the room, more stairs rise to form an
  octagonal dais that also rises above the water. Rusty chains with shackles dangle from the ceiling directly
  above a stone alter mounted on the dais. The alter is carved with hideous depictions of gasping ghouls and
  is stained with dry blood.
  
  There is a small, white bundle visible atop the altar.
\end{readout}
The water is 2 feet deep. The ledges and central dais are 5 feet high (3 feet higher than the water's surface),
and the chamber's ceiling is 16 feet high (11 feet above the dais and ledges). The chains dangling from the
ceiling are 8 feet long; the cultists would shackle prisoners to the chains, dangle them above the altar, cut
them open with knives, and allow the altar to be bathed in blood.

\subsubsection*{``One Must Die!''}
If any PC climbs to the top of the dais:
\begin{readout}
  The chanting rises once more as thirteen dark apparitions appear on the ledges overlooking the room. Each one
  resembles a black-robed figure holding a torch, but the torche's fire is black and seems to draw light to it.
  Where you'd expect to see faces are voids.
  
  ``One must die!'' they chant, over and over. ``One must die! One must die!''
\end{readout}
The apparitions are harmless figments that can't be damaged, turned, or dispelled.

PCs can ascertain what must be done with a successful DC 11 Intelligence (Religion) or Wisdom (Insight) check.
To count as a sacrifice, a creature must die on the altar. The apparitions don't care what kind of creature
is sacrificed and they aren't fooled by illusions. Remember that, if the players befriended it, the dog Lancelot
counts as a valid sacrifice.

If the PCs refuse to make the requested sacrifice, the cult is angered and summons Walter. If the PCs make the
requested sacrifice, the cult chants victoriously and summons Walter anyway. Either way, your players should
feel as though they have just made a grave error.
\begin{readout}
  Suddenly, the portcullis slams back down behind you! The cultists' chant changes: ``The end comes! Death, be
  praised!'' The dirty water filling the chamber ripples as something moves beneath the surface.
  A host of bones, flesh, and disparate body parts -- some from the refuse pile in the alcove, some from
  under the water -- come together and collect into a massive, shifting heap of gore.
\end{readout}
When the Dark Power accepted Mrs. Durst's final sacrifice, Walter was transformed into a terrible monster: a 
vessel for the cult's hatred, arrogance, and depravity bound within an innocent babe.

Once a PC has seen or learned of Walter's existence, if that PC is aware of the circumstances of Walter's birth
and death, that PC may make a DC 15 Intelligence (Arcana) or DC 15 Intelligence (Religion) check to learn the
source of the curse upon Death House. If they succeed they learn:
\begin{readout}
  The spirit of a murdered infant, unwanted by a parent, can incite a powerful curse upon its household, 
  tormenting its killers and chaining their souls to the place of its death. The only way to remove the curse
  upon the house is to bury Walter's corpse at sunrise beneath the threshold of the dwelling.
\end{readout}
When the PCs defeat or flee from the Flesh Mound, the house responds in kind.
\begin{readout}
  The floor begins to quake, and the ceiling shudders and cracks as debris and dust begin to sift into the air.
  The structure groaning above your heads makes one thing clear - the house is starting to come down around you!
  Fly, you fools!
\end{readout}

\begin{arealinks}
  \arealink{sec:Prison}
  \arealink{sec:Portcullis}
  \item \nameref{sec:SkillChallenge}
\end{arealinks}

\subsubsection*{Skill Challenge Escape: Ritual Chamber}
\label{sec:SC_RitualChamber}
As the PCs move to flee and if it isn't already closed, the portcullis slams shut. On a failure, the PCs
wallow in indecision or struggle to force it open, eventually escaping at the cost of 1 failed check.
\begin{skillChallenge}
  \begin{itemize}
    \item \textbf{Athletics} can be used to force open the portcullis. \moderateDC
    \item \textbf{Insight} or \textbf{Investigation} can be used to recall or rationalize that the nearby
    corridor to the Prison (Area 36) might have a secret door. \hardDC
    \item \textbf{Perception} can be used to spot the hidden door to the Prison (Area 36) providing another
    means of escape from the chamber. \moderateDC
  \end{itemize}
\end{skillChallenge}

\pagebreak
\subsection{Attic Hall}
\label{sec:AtticHall}
\begin{readout}
  This bare hall is choked with dust and cobwebs. The door to the northeast has a padlock on it.
\end{readout}

\subsubsection*{Locked Door}
The door to the children's room (Area 20) is held shut with a padlock. Its key is kept in the library (Area 8),
but the lock can also be picked with thieves' tools and a successful DC 15 Dexterity check or broken by smashing
it with a bludgeoning or slashing weapon and succeeding on a DC 20 Strength check.

\begin{arealinks}
  \arealink{sec:Balcony}
  \arealink{sec:NursemaidsSuite}
  \arealink{sec:SpareBedroomWest}
  \arealink{sec:StorageRoom4tFlr}
  \arealink{sec:SpareBedroomEast}
  \arealink{sec:ChildrensRoom}
\end{arealinks}


\pagebreak
\subsection{Spare Bedroom (West)}
\label{sec:SpareBedroomWest}
\begin{readout}
  This dust-choked room contains a slender bed, nightstand, a small iron stove, a writing desk with a stool,
  an empty wardrobe, and a rocking chair. A smiling doll in a lacy yellow dress sits in the northern window
  box, cobwebs draping it like a wedding veil.
\end{readout}

\begin{arealinks}
  \arealink{sec:AtticHall}
\end{arealinks}


\pagebreak
\subsection{Storage Room (Fourth Floor)}
\label{sec:StorageRoom4tFlr}
\begin{readout}
  This dusty chamber is packed with old furniture -- chairs, coat racks, standing mirrors, dress mannequins,
  and the like -- all draped in dusty white sheets. Near an iron stove, underneath one of the sheets, is
  a wooden trunk. The far wall has a window set in an alcove, through which you can see occasional flashes
  of lightning.
\end{readout}
If a PC opens the trunk:
\begin{readout}
  The trunk is unlocked and contains skeletal remains of what appears to be a young woman. She is wrapped in
  a tattered bedsheet stained with dry blood.
\end{readout}
PCs with proficiency in medicine / a DC 14 Wisdom (Medicine) check can verify that the woman was stabbed
to death by multiple knife wounds. If the PCs disturb the remains in a disrespectful fashion, the nursemaid's
specter appears and attacks unless it was previously defeated in the nursemaid's suite (Area 15).

\subsubsection*{Secret Door}
A secret door in the east wall appears only when certain conditions are met; see the secret stairs (Area 21)
for more information.

\begin{arealinks}
  \arealink{sec:AtticHall}
  \arealink{sec:SecretStairs}
\end{arealinks}


\pagebreak
\subsection{Spare Bedroom (East)}
\label{sec:SpareBedroomEast}
\begin{readout}
  This web-filled room contains a slender bed, a nightstand, a rocking chair, an empty wardrobe, and a small
  iron stove.
\end{readout}

\begin{arealinks}
  \arealink{sec:AtticHall}
\end{arealinks}


\pagebreak
\subsection{Children's Room}
\label{sec:ChildrensRoom}
The door to this room is locked from the outside (see the attic hall (Area 16) for details).
\begin{readout}
  This room contains a bricked-up window flanked by two dusty, wood-framed beds sized for children. Closer
  to the door is a toychest with windmills painted on its sides and a dollhouse that appears to be a perfect
  replica of the dreary edifice in which you stand. These furnishings are draped in cobwebs. Lying in the
  middle of the floor are two small skeletons wearing tattered but familiar clothing, and the smaller of the
  two is cradling an also-familiar stuffed doll.
\end{readout}
The Durst children, Rose and Thorn, were neglected by their parents and locked in this room until they starved
to death. Their small skeletons lie in the middle of the floor, plain as day, wearing tattered clothing that
the PCs recognize as belonging to the children. Thorn's skeleton cradles the boy's stuffed doll.

The toychest contains an assortment of stuffed animals and toys. PCs who search the dollhouse (and succeed
on a DC 15 Wisdom (Perception) check if pressed for time) find all of the house's secret doors, including
one in the attic that leads to a spiral staircase (a miniature replica of Area 21).

\subsubsection*{Rose and Thorn}
If either the dollhouse or the chest is disturbed, the ghosts of Rose and Thorn appear in the middle of
the room. Use the ghost statistics in the Monster Manual, with the following modifications:
\begin{itemize}
  \item The ghosts are small and lawful good.
  \item They have 35 (10d6) hit points each.
  \item They lack the Horrifying Visage action.
  \item They speak Common and have a challenge rating of 3 (700 XP).
\end{itemize}
The children don't like it when they PCs disturb their toys, but they fight only in self-defense. Unlike the
illusions outside the house, these children know that they're dead.

If asked how they died, Rose and Thorn explain that their parents locked them in the attic to protect
them from ``the monster in the basement'', and that they died from hunger.

If asked how one gets to the basement, Rose points to the dollhouse and says ``There's a secret door in
the attic''. PCs who then search the dollhouse for secret doors automatically find them (or gain advantage
on their Wisdom (Perception) checks to find them if pressed for time).

The children fear abandonment. If one or more PCs try to leave, the ghost-children attempt to possess them.
If one of the ghosts possesses a PC, allow the player to retain control of the PC but assign one of the
following flaws:
\begin{itemize}
  \item Possessed by Rose: ``I like being in charge and get angry when other people tell me what to do.''
  \item Possessed by Thorn: ``I'm scared of everything, including my own shadow, and weep with despair when
  things don't go my way.''
\end{itemize}
The possession can be ended either by laying their bones to rest or through a successful DC 11 Charisma (Intimidation/Pursuasion) check. A ghost reduced to 0 hit points can reform at dawn the next day.

Take care to play the encounter with Rose and Thorn's ghosts as humanizing and sympathetic, rather than
alien and frightening. Despite their nature as centuries-old spirits, Rose and Thorn are fundamentally
children -- lost, scared, yet innocent children. As such, when Rose or Thorn attempt to possess a PC,
try to present it in such a manner that the PC willingly allows the spirit to enter their body. If you
describe it as ``a child's tiny hand, desperately seeking the warm embrace of another soul'', your players
may even decline to roll a saving throw against possession. A PC that is possessed by Rose or Thorn can
continue to communicate with the child's spirit as a voice in their head.

If you would like to deepen Rose's characterization, you may make her a child prodigy who is quite aware
of the dynamics in her house (though not the specific details). She is also a budding wizard who discovered
a small spellbook in her father's library, and took great care in copying the Mending, Light, and Shocking
Grasp cantrips into her diary. To demonstrate this, you may choose to have the ghostly Thorn accidentally
break one of his torys, which Rose swiftly Mends. Should she possess a PC, she is able to cast those cantrips
through her host's body.

Rose shyly shares her diary with the party if her use of magic is remarked upon.
\begin{readout}
  Rose's diary contains entries regarding her studies, her friends, her younger brother (who she is fiercely
  protective of), and elementary (yet insightful) observations on the nature of magic. The Mending, Light, and
  Shocking Grasp cantrips have, clearly with great care, been copied into the diary. Just below Rose's notes
  on Shocking Grasp there is another scribbled comment that reads: ``It worked! Uncle Dimov snuck into our room
  again, but this time I was ready. I hope he never comes back!''
\end{readout}
If the PCs attempt to discuss this incident or Uncle Dimov with Rose, she instantly clams up. If Thorn is
asked about his uncle he shrinks in on himself and falls silent, with Rose hugging him while glaring daggers
at the PC responsible. If you would prefer to avoid the implication of child abuse, you may have Rose instead
explain that Uncle Dimov would break Thorn's toys and taunt him for his timid nature and weak constitution.

Once the PCs have made friends or allies of Rose and Thorn, if the PCs appear wounded or tired the ghostly
children offer the use of their room as a sanctuary for a rest and promise to stand guard against the
``monsters'' that they've heard below.

While ignorant to the tru nature of the cult, Rose remembers hearing her mother bringing Walter to the
basement before she [Rose] died. She asks the PCs to save their baby brother and parents, and defeat the
monster below once and for all.

Rose knows the way down to the basement, but ``isn't supposed to go down there'' and ``doesn't want to
get in trouble''. If the party convinces her to show them the way, she points them toward the dollhouse,
revealing the secret entry. In exchange, she asks the PCs to deliver her and Thorn's bones to their
resting places in the crypts below.

The dollhouse contains small dolls that depict tiny, twisted molds of any characters and creatures currently
visible in the house. The dolls are made of painted resin. Any PC looking inside the dollhouse while in Rose
and Thorn's room can see the appropriately-placed dolls of all living creatures within the manor.

\subsubsection*{Development}
If the party lays the children's spirits to rest, each character gains inspiration (see ``Inspiration'' in
chapter 4, ``Personality and Background'', of the Player's Handbook).

\begin{arealinks}
  \arealink{sec:AtticHall}
\end{arealinks}


\pagebreak
\subsection{Secret Stairs}
\label{sec:SecretStairs}
\uline{Make sure to remind your players to prepare to reach level two before the session that you expect
them to find this area.}
\begin{readout}
  The door opens to a shaft of mortared stone containing a narrow spiral staircase made of creaky wood.
  Thick cobwebs fill the shaft, greatly reducing your visibilty to about 5 feet as you descend.
\end{readout}
The secret door and shaft don't exist until the house reveals them, which can happen in one of two ways:
\begin{itemize}
  \item PCs find Strahd's letter in the secret room behind the library (Area 9).
  \item PCs find the replica secret door in the attic of the dollhouse (Area 20).
\end{itemize}
Once the house wills the secret door into existence, PCs find it automatically if they search the wall
(no ability check required). PCs who descend the spiral staircase end up in the dungeon level access (Area 22).

\begin{arealinks}
  \arealink{sec:StorageRoom4tFlr}
  \arealink{sec:DungeonLevelAccess}
\end{arealinks}

\subsubsection*{Skill Challenge Escape: Secret Stairs}
\label{sec:SC_SecretStaircase}
\begin{readout}
  The mortared walls of this staircase have been covered with swarms of infant spiders. As you flee up the
  steps, an enormous giant spider climbs up from the depths and attempts to start dragging [\textit{the last
  PC in marching order}] back down the steps.
\end{readout}
A PC that fails here loses time struggling against the spider.
\begin{skillChallenge}
  \begin{itemize}
    \item \textbf{Animal Handling} or \textbf{Intimidation} can be used to scare off the spider, especially if
    fire is used or an attack is made. \moderateDC
    \item \textbf{Athletics} or \textit{Acrobatics} can be used to wrench the PC out of the web or to keep the 
    spider from dragging the PC away. \moderateDC
    \item \textbf{Levelled Spells} (Automatic Success) or \textbf{cantrips} \easyDC can be cast to subdue the
    spider or burn the web (e.g. \textit{animal friendship}, \textit{produce flame}, and \textit{firebolt})
  \end{itemize}
\end{skillChallenge}

\pagebreak
\subsection{Balcony}
\label{sec:Balcony}
\begin{readout}
  At the top of the stairs you arrive at a dusty balcony, and if you look over the railing of the spiraling
  staircase you're able to see all the way down to the first floor through the center.
  
  There is a suit of black plate armor standing against one wall, draped in cobwebs. Oil lamps are mounted
  on the oak-paneled walls, which are carved with woodland scenes of trees, falling leaves, and tiny critters.
  \{\textit{PCs with 12+ passive perception:} You notice tiny corpses hanging from the trees and worms bursting
  up from the ground.\}
\end{readout}

If PCs had played the harpsichord in the conservatory (Area 10):
\begin{readout}
  It looks like the entire stone wall swung open at the western end of the landing to reveal a secret doorway.
\end{readout}

This suit of animated armor attacks as soon as it takes damage or a PC approaches within 5 feet of it. It fights
until destroyed. If the animated armor is thrown down to the first floor and the PCs don't reveal their
presence atop the balcony, it is unable to observe them with its sixty feet of blindsight and is too stupid
to think to climb back up.

A creature that is pushed over the edge of the balcony falls two stories, or twenty feet, and takes 2d6
bludgeoning damage. That creature must succeed on a DC 15 Dexterity (Acrobatics) check or land prone.

\subsubsection*{Know the Monsters: Animated Armor}
The armor will attempt to push PCs over the railing itself using a shove attack, or attempt to grapple its 
nearest target before shoving them prone. Remember that, as the armor has two attacks per round, it is able to
make two shove or grapple attempts each turn -- or any mixture of the two.

\subsubsection*{Secret Door}
A secret door in the west wall can be found with a successful DC 15 Wisdom (Perception) check. It pushes open
easily to reveal a cobweb-filled wooden staircase leading up to the attic.

\begin{arealinks}
  \arealink{sec:UpperHall}
  \arealink{sec:MasterSuite}
  \arealink{sec:Bathroom}
  \arealink{sec:StorageRoom3dFlr}
  \arealink{sec:NursemaidsSuite}
  \arealink{sec:AtticHall}
\end{arealinks}


\pagebreak
\subsection{Master Suite}
\label{sec:MasterSuite}
\begin{readout}
  The double doors to this room have dusty panes of stained glass set into them; the designs in the glass
  resemble windmills. The dusty, cobweb-filled master bedroom has burgundy drapes covering the windows.
  Furnishings include a four-poster bed with embroidered curtains and tattered gossamer veils, a matching pair
  of empty wardrobes, a vanity with a wood-framed mirror and jewelry box, and a padded chair. A rotting
  tiger-skin rug lies on the floor in front of the fireplace, which has a dust-covered portrait hanging above it
  of \{\textit{If PCs have found the will:} Gustav and Elisabeth Durst / \textit{Otherwise:} the same dour
  man and woman you've seen pictured throughout the house\}. There is a male ghast dressed in fine clothes
  hanging from a noose tied beside the bed, and he has a small piece of parchment clutched in his right hand.
\end{readout}
As PCs move into the room in general:
\begin{readout}
  A web-filled parlor in the southwest corner contains a table and two chairs. Resting on the dusty tablecloth
  is an empty porcelain bowl and a matching rug. A door facing the foot of the bed has a full-length mirror
  mounted on it, and there is a door behind one of the chairs in the parlor. Additionally, in the back corner
  of the parlor you notice a small, 2-foot-wide door set into the wall with a button next to it.
\end{readout}
The door facing the foot of the bed (behind the mirror) opens to reveal an empty, dust-choked closet (Area 12B).
A door in the parlor leads to an outside balcony (Area 12C).

As PCs approach the ghast:
\begin{readout}
  The hanging body doesn't move as you approach -- it's clearly dead. \{\textit{PCs with proficiency
  in medicine / a DC 13 Wisdom (Medicine) check:} The body has been dead for no more than a few hours.\}
  The note is held tight by rigor mortis, but you're able to wiggle it free.
\end{readout}
The body is Mr. Durst. He is dead, and does not attack the PCs when disturbed. The appearance of only being
dead a few hours is another manifestation of the house's curse, and not actually correct.
The letter clutched in his hand reads:
\begin{handout}
  My Beloved Children,
  
  I wish I could be what all fathers do and tell you that monsters aren't real. But it wouldn't be true.
  
  Life can create things of exquisite beauty. But it can also twist them into hideous beings. Selfish. Violent.
  Grotesque. Monstrous. It hurts me to say that your mother has turned into one such monster, inside and out.
  And I'm afraid the disease that afflicted her mind has taken hold of me as well.
  
  It sickens me to think what we've put you through. There is no excuse. I only ask you, though I know I have
  not the right to do so, to try and forgive us. I despise what your mother has become, but I love and pity
  her all the same.
  
  Rose, I wish I could see you blossom into a strong, beautiful woman. Thorn and Walter, I wish I could be
  there for you. But I can't. This is the only way.
  
  Goodbye.
\end{handout}

\subsubsection*{Dumbwaiter}
A dumbwaiter in the corner of the west wall has a button on the wall next to it. Pressing the button rings a
tiny bell in the kitchen (Area 4A).

\subsubsection*{Treasure}
\begin{readout}
  The jewelry box on the vanity is made of silver with gold filigree (worth 75 gp). It contains three gold rings
  (worth 25 gp each), a thin platinum necklace with a topaz pendant (worth 750 gp), and an unsent letter
  addressed to one Mrs. Petrovna.
\end{readout}
The letter is from Mrs. Durst to a fellow cult member. It reads:
\begin{handout}
  My Dear Mrs. Petrovna,
  
  Your advice on dealing with the unwanted fiend in my home is quite good advice indeed. Tonight's ceremony
  will proceed as planned when the moon is at its highest peak -- without, of course, the attendance of
  Mr. Durst. I must agree with you that, with the assistance of such a remarkably innocent subject, the results
  of our proceedings may be far improved. ``Innocent'', of course, is not quite the term I would use.
  
  If nothing else, I am relieved that I shall soon no longer have to suffer the harlot's insufferable presence
  each time we must pass through her quarters to our meeting-space. We shall be well rid of her indeed.
  
  My thanks,
  
  Mrs. Elisabeth Durst
\end{handout}

\begin{arealinks}
  \arealink{sec:Balcony}
\end{arealinks}


\pagebreak
\subsection{Bathroom}
\label{sec:Bathroom}
\begin{readout}
  This dark room contains a wooden tub with clawed feet, a small iron stove with a kettle resting atop it,
  and a barrel under a spigot in the east wall. There is a pipe coming through the ceiling which feeds into
  the spigot.
\end{readout}
A cistern on the roof used to collect rainwater, which was borne down a pipe to the spigot; however, the
plumbing no longer works.

\begin{arealinks}
  \arealink{sec:Balcony}
\end{arealinks}


\pagebreak
\subsection{Storage Room (Third Floor)}
\label{sec:StorageRoom3dFlr}
\begin{readout}
  Dusty shelves line the walls of this room. A few of the shelves have folded sheets, blankets, and old bars
  of soap on them, and leaning against the far wall is a cobweb-covered broom.
\end{readout}
The broom leaning against the far wall is a broom of animated attack; it attacks any creature approaching
within 5 feet of it.

\begin{arealinks}
  \arealink{sec:Balcony}
\end{arealinks}


\pagebreak
\subsection{Nursemaid's Suite}
\label{sec:NursemaidsSuite}
\begin{readout}
  Dust and cobwebs shroud an elegantly appointed bedroom. There is a large bed, two end tables, and an
  empty wardrobe. Mounted on the wall next to the wardrobe is a full-length mirror with an ornate wooden
  frame carved to look like ivy and berries. \{\textit{PCs with 12+ passive perception:} You notice there
  are eyeballs among the berries.\} There is a small table beside the bed, and atop it is a cobwebbed book
  which appears to have been clumsily hidden beneath a mildew-covered towel.
\end{readout}
Double doors set with panes of stained glass pull open to reveal a balcony (Area 15C) overlooking the front
house.

The bedroom once belonged to the family's nursemaid. The master of the house and the nursemaid had an affair,
which led to the birth of a bastard baby named Walter. The cult slew the nursemaid shortly thereafter.

If a PC investigates the book:
\begin{readout}
  The book appears to be a raunchy romance novel titled ``Blue-Blooded Lips''. It tells the story of a wealthy
  duke who enjoys an affair with his female cupbearer.
\end{readout}
If a PC opens the door to the nursery (Area 15B):
\begin{readout}
  There is a crib covered with a hanging black shroud. \{\textit{Ignore this if PCs have already defeated
  specter in 3rd-floor storage room (Area 18):}\} Standing beside the crib is a specter in the form
  of a young woman; she is wearing a homely dress and bonnet, and is staring gently into a crib.
  She doesn't react to you opening the door.
\end{readout}
When PCs part the shroud covering the crib, they see a tightly wrapped, baby-sized bundle lying in the crib.
PCs who unwrap the blanket find nothing inside it.

\subsubsection*{Secret Door}
When any character inspects the mirror, they find the secret door automatically without any checks required.
It makes little sense for such a mundane door to be concealed so expertly, and a DC 15 Wisdom (Perception)
barrier can create undue frustration in your players, given that finding this door is necessary to continue
forward. It pushes open easily to reveal a cobweb-filled wooden staircase leading up to the attic (Area 16).

\subsubsection*{Nursemaid Encounter}
If the nursemaid's specter is disturbed (spoken to, touched, etc.):
\begin{readout}
  The specter turns toward the PCs, holds a finger to her lips, and whispers ``Hush, the baby is sleeping''.
\end{readout}
If a PC threatens or approaches the bundle containing the ``baby'', the specter attacks, relenting only when
all PCs have fled her chambers or when the character that disturbed her ``baby'' has been killed or knocked
unconscious. If her HP gets very low she vanishes in order to reappear later in the 3rd-floor storage room
(Area 18).

If the PCs approach with kindness or respect, the nursemaid's specter introduces herself as Margaret. She
is withdrawn and shy, and does not fully understand how or why she died. She is confused, and frequently
jumps between awareness and ignorance of her own state of undeath, sometimes in the same sentence.

She speaks fondly of Mr. Durst, but avoids their affair out of a sense of propriety. If a PC asks her about
her relationship with Mr. Durst or her parentage of Walter, she smiles sadly and informs the party that it
is ``not her place to speak of such things''. She adores Rose, Thorn, and Walter. While she does not speak
ill of Mrs. Durst if asked, she is clearly uncomfortable and fearful of the lady of the house.

When the PCs finish their conversation:
\begin{readout}
  The nursemaid scoops up the baby in her arms and turns to leave, saying: ``I'm taking Walter upstairs to
  play with his older brother and sister. Take care!'' She smiles warmly and walks into the main bedroom,
  before turning and passing through the mirror on the wall. A moment later, you hear a click and the mirror
  gently swings inward revealing a hidden passageway.
\end{readout}
Margaret does not appear again.

\begin{arealinks}
  \arealink{sec:Balcony}
  \arealink{sec:AtticHall}
\end{arealinks}
